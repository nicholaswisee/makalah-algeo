\documentclass[conference]{IEEEtran}
\IEEEoverridecommandlockouts
% The preceding line is only needed to identify funding in the first footnote. If that is unneeded, please comment it out.
%Template version as of 6/27/2024

\usepackage{cite}
\usepackage{amsmath,amssymb,amsfonts}
\usepackage{algorithmic}
\usepackage{graphicx}
\usepackage{textcomp}
\usepackage{xcolor}
\usepackage{hyperref}
\def\BibTeX{{\rm B\kern-.05em{\sc i\kern-.025em b}\kern-.08em
    T\kern-.1667em\lower.7ex\hbox{E}\kern-.125emX}}
\begin{document}

\title{Analysis of Steady-State Behavior in Server Queues using Markov Chains and Eigenvalues in the M/M/1 Model\\}

\author{\IEEEauthorblockN{Nicholas Wise Saragih Sumbayak - 13524037}
\IEEEauthorblockA{\textit{School of Electrical Engineering and Informatics} \\
\textit{Institut Teknologi Bandung}\\
Jl. Ganesha 10 Bandung 40132, Indonesia \\
\href{mailto:nicholasaragih@gmail.com}{nicholasaragih@gmail.com} | \href{mailto:13524037@std.stei.itb.ac.id}{13524037@std.stei.itb.ac.id}}
}

\maketitle

\begin{abstract}
This document is a model and instructions for \LaTeX.
This and the IEEEtran.cls file define the components of your paper [title, text, heads, etc.]. *CRITICAL: Do Not Use Symbols, Special Characters, Footnotes, 
or Math in Paper Title or Abstract.
\end{abstract}

\begin{IEEEkeywords}
component, formatting, style, styling, insert.
\end{IEEEkeywords}

\section{Introduction}
Queueing behavior arises naturally in nearly every computing environment
where resources are shared among multiple tasks. Whenever incoming work arrives 
faster than it can be immediately processed, the excess work must wait, forming
a queue. This phenomenon appears in a broad range of systems, including CPU 
scheduling, packet forwarding in routers, job dispatching in cloud infrastructures,
disk I/O scheduling, and networked application servers. The performance of 
these systems is heavily influenced by their queueing characteristics, making analytical
models essential for understanding and improving real-world performance \cite{b1}.

Modern computing workloads are highly variable and unpredictable. 
Task arrivals do not occur at fixed intervals, and service times fluctuate 
due to user behavior, network delays, resource contention, and software-level
scheduling. These uncertainties make deterministic approaches nonoptimal. 
However, probabilistic theory enables engineers to determine whether a server will remain stable 
under a particular load, estimate average waiting times, and understand how 
performance degrades as traffic increases.

To formally analyze such behavior, queueing systems are commonly modeled as 
stochastic processes, with one of the simplest and most fundamental being the M/M/1 queue. 
In this model, arrivals follow a Poisson process with service times that follow an exponential
distribution, all handled by a single server. Despite its simplicity, the model can capture
trends and predict behaviors such as stability, queue buildup, and performance spikes. 
\cite{b2}.

The M/M/1 queue can be naturally represented as a Markov chain, where each state corresponds 
to the current number of jobs in the system. Transitions between states reflect arrivals and 
completions of jobs, producing a stochastic matrix representing the dynamic behavior of 
the queue. The steady-state distribution corresponds to the eigenvector 
associated with eigenvalue 1, consistent with standard results in Markov chain theory \cite{b3}. 
This connection between queueing systems and linear algebra provides a powerful framework for 
analyzing server load and predicting performance under varying traffic conditions.

In this paper, we examine the steady-state behavior of the M/M/1 queue using Markov chains 
and eigenvalue analysis. We first derive the transition matrix of the system and compute 
its stationary distribution using the eigenvector method. We then compare this theoretical 
distribution with the known closed-form solution of the M/M/1 queue. To validate the  
results, a queue simulation is implemented in C++, allowing us to observe empirical queue 
behavior under different arrival and service rates. By comparing theoretical predictions with 
simulation outcomes, we illustrate how mathematical modeling can determine whether a server 
can withstand a particular load and how queue lengths evolve over time.

\section{Theoretical Foundation}

\subsection{Matrices}
\subsubsection{Definition}
A matrix is a rectangular array of numbers, symbols, or expressions, arranged in rows and columns.
The individual items in a matrix are called its elements or entries. The size of a matrix is described
in terms of the number of rows and columns it contains. Generally, a general $m \times n$ matrix 
may be denoted as:

\[
A = 
\begin{bmatrix}
a_{11} & a_{12} & \cdots & a_{1n} \\
a_{21} & a_{22} & \cdots & a_{2n} \\
\vdots & \vdots & \ddots & \vdots \\
a_{m1} & a_{m2} & \cdots & a_{mn} \\
\end{bmatrix}
\]

with A denoting the matrix, m the number of rows, n the number of columns, and $a_{ij}$ the element
in row $i$ and column $j$. A matrix with size $n \times n$ is called a square matrix of order n and
the elements $a_{ii}$ (where the row and column indices are equal) form the main diagonal
of matrix A.

\subsubsection{Row and Column Vectors}
A matrix with only one row or one column is called a row matrix (or a row vector) or column matrix 
(or a column vector), respectively. A general $m \times 1$ column matrix and a $1 \times n$ row matrix b may be denoted as:

\begin{center}
    $\textbf{a} = \begin{bmatrix} a_{1} \\ a_{2} \\ \vdots \\ a_{m} \end{bmatrix}$ and $\textbf{b} = \begin{bmatrix} b_{1} & b_{2} & \cdots & b_{n} \end{bmatrix}$
\end{center}

\subsubsection{Matrix Addition and Subtraction}
If A and B are matrices of the same size, their sum (A + B) and difference (A - B) are obtained by adding or 
subtracting their corresponding entries. Matrices of different sizes cannot be added or subtracted. In matrix notation,
if $A = [a_{ij}]$ and $B = [b_{ij}]$ have the same size, then $(A \pm B)_{ij} = a_{ij} \pm b_{ij}$. 

\subsubsection{Scalar Multiplication}
If A is any matrix and k is any scalar, then the scalar multiple kA is the matrix obtained by multiplying every entry of A by k.
In matrix notation, if $A = [a_{ij}]$, then $(kA)_{ij} = k a_{ij}$.

\begin{center}
$A = \begin{bmatrix} 1 & 2 & 3 \\ 4 & 5 & 6 \end{bmatrix} \rightarrow 2A = \begin{bmatrix} 2 & 4 & 6 \\ 8 & 10 & 12 \end{bmatrix}$    
\end{center}

\subsubsection{Matrix Multiplication}
If A is an $m \times n$ matrix and B is an $n \times p$ matrix, then the product AB is defined to be the $m \times p$ matrix C
whose entries are given by:
\[c_{ij} = \sum_{k=1}^{n} a_{ik} b_{kj}\]

In simpler terms, to find entry $c_{ij}$ of the product matrix C, multiply the corresponding entries of the $i^{th}$ row of 
matrix A with the $j^{th}$ column of matrix B and add the results. Given the example below:

\begin{center}
$A = \begin{bmatrix} 1 & 2 & 3 \\ 3 & 4 & 0 \end{bmatrix}$ , $B = \begin{bmatrix} 2 & 1 \\ 4 & 3 \\ 5 & 0 \end{bmatrix}$
\end{center}

\begin{center}
$C = AB = \begin{bmatrix} 25 & 7 \\ 22 & 15 \end{bmatrix}$
\end{center}

Since A is a $2 \times 3$ matrix and B is a $3 \times 2$ matrix, the resulting product AB is a $2 \times 2$ matrix. For example,
to determine entry $c_{11}$ of the product matrix AB, we multiply the corresponding entries of the first row of matrix A 
with the first column of matrix B and add the results.

\subsubsection{Transpose of a Matrix}
Given a matrix $A$ of size $m \times n$, the transpose of $A$, denoted by $A^T$, is the $n \times m$ matrix obtained 
by interchanging the rows and columns of A. In matrix notation, if $B = A^T$, then the entries of B are defined as
$b_{ij} = a_{ji}$, $1 \leq i \leq n$, $1 \leq j \leq m$.
\begin{center}
$A = \begin{bmatrix} 1 & 2 & 3 \\ 4 & 5 & 6 \end{bmatrix} \rightarrow A^T = \begin{bmatrix} 1 & 4 \\ 2 & 5 \\ 3 & 6 \end{bmatrix}$
\end{center}

\subsection{Eigenvalues and Eigenvectors}

If A is an $n \times n$ matrix, then a nonzero vector $\textbf{v}$ in $R^n$ is called an eigenvector of A if there exists a scalar 
$\lambda$ such that:

\begin{center}
$A \textbf{v} = \lambda \textbf{v}$
\end{center}

The scalar $\lambda$ is called the eigenvalue of A corresponding to the eigenvector $\textbf{v}$. In other words, multiplying 
the matrix A by the vector $\textbf{v}$ results in a new vector that is a scalar multiple of the original vector $\textbf{v}$.

Given a matrix A with size $n \times n$, the eigenvalues and eigenvectors are found by solving the 
following characteristic equation:
\begin{center}
$Ax = \lambda x$ \\
$IAx = \lambda Ix$ \\
$Ax = \lambda Ix$ \\
$(A - \lambda I)x = 0$ \\
\end{center}

Since $x = 0$ is the only trivial solution, for $(A - \lambda I)x = 0$ to have non-trivial solutions, 
the matrix $(A - \lambda I)$ must be singular, and therefore $det(A - \lambda I)$ must be zero.
The polynomial given by $det(A - \lambda I) = 0$ is called the characteristic equation of A, and the solutions
to such equation are the eigenvalues of A, otherwise denoted as the characteristic roots.


\subsection{Markov Chains}
A markov chain is a mathematical system used to model systems that transition between different states over time.
The defining characteristic of a markov chain is that the probability of transitioning to the next state depends only
on the current state, and not on the sequence of states that preceded it. This property is known as the markov property.

A markov chain is commonly represented using a matrix formed by probability vectors which represent the likelihood of the system 
transitioning from one state to another. This matrix is called the transition matrix and the state vectors at successive time 
intervals can are defined by  $x(n+1) = P x(n)$, where $P$ is the transition matrix and $P_{ij}$ is the probability that the system
will be in state $i$ at time $n+1$ given that it was in state $j$ at time $n$.

\begin{figure}[htbp]
\centerline{\includegraphics[width=\columnwidth]{Markov-Chains-in-NLP.png}}
\caption{Example of a Markov Chain representing weather states \cite{b6}.}
\label{fig:markov}
\end{figure}


\subsection{Equations}
Number equations consecutively. To make your 
equations more compact, you may use the solidus (~/~), the exp function, or 
appropriate exponents. Italicize Roman symbols for quantities and variables, 
but not Greek symbols. Use a long dash rather than a hyphen for a minus 
sign. Punctuate equations with commas or periods when they are part of a 
sentence, as in:
\begin{equation}
a+b=\gamma\label{eq}
\end{equation}

Be sure that the 
symbols in your equation have been defined before or immediately following 
the equation. Use ``\eqref{eq}'', not ``Eq.~\eqref{eq}'' or ``equation \eqref{eq}'', except at 
the beginning of a sentence: ``Equation \eqref{eq} is . . .''

\subsection{\LaTeX-Specific Advice}

Please use ``soft'' (e.g., \verb|\eqref{Eq}|) cross references instead
of ``hard'' references (e.g., \verb|(1)|). That will make it possible
to combine sections, add equations, or change the order of figures or
citations without having to go through the file line by line.

Please don't use the \verb|{eqnarray}| equation environment. Use
\verb|{align}| or \verb|{IEEEeqnarray}| instead. The \verb|{eqnarray}|
environment leaves unsightly spaces around relation symbols.

Please note that the \verb|{subequations}| env ironment in {\LaTeX}
will increment the main equation counter even when there are no
equation numbers displayed. If you forget that, you might write an
article in which the equation numbers skip from (17) to (20), causing
the copy editors to wonder if you've discovered a new method of
counting.

{\BibTeX} does not work by magic. It doesn't get the bibliographic
data from thin air but from .bib files. If you use {\BibTeX} to produce a
bibliography you must send the .bib files. 

{\LaTeX} can't read your mind. If you assign the same label to a
subsubsection and a table, you might find that Table I has been cross
referenced as Table IV-B3. 

{\LaTeX} does not have precognitive abilities. If you put a
\verb|\label| command before the command that updates the counter it's
supposed to be using, the label will pick up the last counter to be
cross referenced instead. In particular, a \verb|\label| command
should not go before the caption of a figure or a table.

Do not use \verb|\nonumber| inside the \verb|{array}| environment. It
will not stop equation numbers inside \verb|{array}| (there won't be
any anyway) and it might stop a wanted equation number in the
surrounding equation.

\subsection{Some Common Mistakes}\label{SCM}
\begin{itemize}
\item The word ``data'' is plural, not singular.
\item The subscript for the permeability of vacuum $\mu_{0}$, and other common scientific constants, is zero with subscript formatting, not a lowercase letter ``o''.
\item In American English, commas, semicolons, periods, question and exclamation marks are located within quotation marks only when a complete thought or name is cited, such as a title or full quotation. When quotation marks are used, instead of a bold or italic typeface, to highlight a word or phrase, punctuation should appear outside of the quotation marks. A parenthetical phrase or statement at the end of a sentence is punctuated outside of the closing parenthesis (like this). (A parenthetical sentence is punctuated within the parentheses.)
\item A graph within a graph is an ``inset'', not an ``insert''. The word alternatively is preferred to the word ``alternately'' (unless you really mean something that alternates).
\item Do not use the word ``essentially'' to mean ``approximately'' or ``effectively''.
\item In your paper title, if the words ``that uses'' can accurately replace the word ``using'', capitalize the ``u''; if not, keep using lower-cased.
\item Be aware of the different meanings of the homophones ``affect'' and ``effect'', ``complement'' and ``compliment'', ``discreet'' and ``discrete'', ``principal'' and ``principle''.
\item Do not confuse ``imply'' and ``infer''.
\item The prefix ``non'' is not a word; it should be joined to the word it modifies, usually without a hyphen.
\item There is no period after the ``et'' in the Latin abbreviation ``et al.''.
\item The abbreviation ``i.e.'' means ``that is'', and the abbreviation ``e.g.'' means ``for example''.
\end{itemize}
An excellent style manual for science writers is \cite{b7}.

\subsection{Authors and Affiliations}\label{AAA}
\textbf{The class file is designed for, but not limited to, six authors.} A 
minimum of one author is required for all conference articles. Author names 
should be listed starting from left to right and then moving down to the 
next line. This is the author sequence that will be used in future citations 
and by indexing services. Names should not be listed in columns nor group by 
affiliation. Please keep your affiliations as succinct as possible (for 
example, do not differentiate among departments of the same organization).

\subsection{Identify the Headings}\label{ITH}
Headings, or heads, are organizational devices that guide the reader through 
your paper. There are two types: component heads and text heads.

Component heads identify the different components of your paper and are not 
topically subordinate to each other. Examples include Acknowledgments and 
References and, for these, the correct style to use is ``Heading 5''. Use 
``figure caption'' for your Figure captions, and ``table head'' for your 
table title. Run-in heads, such as ``Abstract'', will require you to apply a 
style (in this case, italic) in addition to the style provided by the drop 
down menu to differentiate the head from the text.

Text heads organize the topics on a relational, hierarchical basis. For 
example, the paper title is the primary text head because all subsequent 
material relates and elaborates on this one topic. If there are two or more 
sub-topics, the next level head (uppercase Roman numerals) should be used 
and, conversely, if there are not at least two sub-topics, then no subheads 
should be introduced.

\subsection{Figures and Tables}\label{FAT}
\paragraph{Positioning Figures and Tables} Place figures and tables at the top and 
bottom of columns. Avoid placing them in the middle of columns. Large 
figures and tables may span across both columns. Figure captions should be 
below the figures; table heads should appear above the tables. Insert 
figures and tables after they are cited in the text. Use the abbreviation 
``Fig.~\ref{fig}'', even at the beginning of a sentence.

\begin{table}[htbp]
\caption{Table Type Styles}
\begin{center}
\begin{tabular}{|c|c|c|c|}
\hline
\textbf{Table}&\multicolumn{3}{|c|}{\textbf{Table Column Head}} \\
\cline{2-4} 
\textbf{Head} & \textbf{\textit{Table column subhead}}& \textbf{\textit{Subhead}}& \textbf{\textit{Subhead}} \\
\hline
copy& More table copy$^{\mathrm{a}}$& &  \\
\hline
\multicolumn{4}{l}{$^{\mathrm{a}}$Sample of a Table footnote.}
\end{tabular}
\label{tab1}
\end{center}
\end{table}

\begin{figure}[htbp]
\centerline{\includegraphics{fig1.png}}
\caption{Example of a figure caption.}
\label{fig}
\end{figure}

Figure Labels: Use 8 point Times New Roman for Figure labels. Use words 
rather than symbols or abbreviations when writing Figure axis labels to 
avoid confusing the reader. As an example, write the quantity 
``Magnetization'', or ``Magnetization, M'', not just ``M''. If including 
units in the label, present them within parentheses. Do not label axes only 
with units. In the example, write ``Magnetization (A/m)'' or ``Magnetization 
\{A[m(1)]\}'', not just ``A/m''. Do not label axes with a ratio of 
quantities and units. For example, write ``Temperature (K)'', not 
``Temperature/K''.

\section*{Acknowledgment}

The preferred spelling of the word ``acknowledgment'' in America is without 
an ``e'' after the ``g''. Avoid the stilted expression ``one of us (R. B. 
G.) thanks $\ldots$''. Instead, try ``R. B. G. thanks$\ldots$''. Put sponsor 
acknowledgments in the unnumbered footnote on the first page.

\section*{References}

Please number citations consecutively within brackets \cite{b1}. The 
sentence punctuation follows the bracket \cite{b2}. Refer simply to the reference 
number, as in \cite{b3}---do not use ``Ref. \cite{b3}'' or ``reference \cite{b3}'' except at 
the beginning of a sentence: ``Reference \cite{b3} was the first $\ldots$''

Number footnotes separately in superscripts. Place the actual footnote at 
the bottom of the column in which it was cited. Do not put footnotes in the 
abstract or reference list. Use letters for table footnotes.

Unless there are six authors or more give all authors' names; do not use 
``et al.''. Papers that have not been published, even if they have been 
submitted for publication, should be cited as ``unpublished'' \cite{b4}. Papers 
that have been accepted for publication should be cited as ``in press'' \cite{b5}. 
Capitalize only the first word in a paper title, except for proper nouns and 
element symbols.

For papers published in translation journals, please give the English 
citation first, followed by the original foreign-language citation \cite{b6}.

\begin{thebibliography}{00}
\bibitem{b1} M. Harchol-Balter, \textit{Performance Modeling and Design of Computer Systems: Queueing Theory in Action}. Cambridge, U.K.: Cambridge Univ. Press, 2013.

\bibitem{b2} L. Kleinrock, \textit{Queueing Systems, Volume 1: Theory}. New York, NY, USA: Wiley, 1975.

\bibitem{b3} J. R. Norris, \textit{Markov Chains}. Cambridge, U.K.: Cambridge Univ. Press, 1997.

\bibitem{b4} H. Anton and C. Rorres, \textit{Elementary Linear Algebra: Applications Version}, 11th ed. Hoboken, NJ, USA: Wiley, 2013.

\bibitem{b5} Munir, Rinaldi. "Nilai Eigen dan Vektor Eigen (Bagian 1)", School of Electrical Engineering and Informatics (STEI) ITB, 2025. [Online]. Available: \url{https://informatika.stei.itb.ac.id/~rinaldi.munir/AljabarGeometri/2025-2026/Algeo-19-Nilai-Eigen-dan-Vektor-Eigen-Bagian1-2025.pdf}
    
\bibitem{b6} "Markov Chains in NLP," GeeksforGeeks. [Online]. Available: \url{https://www.geeksforgeeks.org/nlp/markov-chains-in-nlp/}. [Accessed: Dec. 16, 2025].

\end{thebibliography}

\vspace{12pt}
\color{red}
IEEE conference templates contain guidance text for composing and formatting conference papers. Please ensure that all template text is removed from your conference paper prior to submission to the conference. Failure to remove the template text from your paper may result in your paper not being published.

\end{document}
